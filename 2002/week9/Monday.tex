\chapter{Week9}

\section{Monday}\index{Monday_lecture}
\subsection{Laplace Transformation}
\begin{definition}
Let $f$ be a piecewise continuous function on $(0,\infty)$ and is of ``exponential'' order, i.e., $|f(t)|\leq Me^{ct}$, where $M, C$ are two constants. Then the \bf{Laplace transformation} of $f$, denoted by $\mathscr{L}\{f\}$, is given by 
\[F(s)=\mathscr{L}\{f\}(s)=\int_0^\infty e^{-st}f(t)\diff t
\]
, for $s>c$


\end{definition}
\begin{remark}
$\mathscr{L}$ is an operator on $f$. $F$ is a function of $s$.
\end{remark}

Properties:
\begin{itemize}
\item $\mathscr{L}$ is linear.
\item $\mathscr{L}\{f\p\}(s)=s\mathscr{L}\{f\}-f(0)$ (Often used)
\item $\mathscr{L}\{f\pp\}(s)=s^2F(s)-sf(0)-f\p(0)$
\item $\mathscr{L}\{e^{at}f(t)\}(s)=F(s-a)$
\end{itemize}
(1) $\mathscr{L}$ is an operator on $f$.
\[\begin{aligned}
\mathscr{L}\{c_1f_1+c_2f_2\}(s)&=\int_0^\infty e^{-st}[c_1f_1(t)+c_2f_2(t)]\diff t\\
&=c_1\int_0^\infty e^{-st}f_1(t)\diff t+c_2\int_0^\infty e^{-st}f_2(t)\diff t\\
&=c_1\mathscr{L}\{f_1\}(s)+c_2\mathscr{L}\{f_2\}(s)
\end{aligned}
\]
(2)
\[\begin{aligned}
\mathscr{L}\{f\p\}(s)&=\int_0^\infty e^{-st}f\p(t)\diff t\\
&=\lim_{A\rightarrow\infty}\int_0^Ae^{-st}\diff f(t)\\
&=\lim_{A\rightarrow\infty}(e^{-st}f(t)\mid_{t=0}^A-\int_0^A(-s)f(t)e^{-st}\diff t)\quad\text{Differentiation by part}\\
&=\lim_{A\rightarrow\infty}e^{-sA}f(A)-f(0)+s\lim_{A\rightarrow\infty}\int_0^Af(t)e^{-st}\diff t\\
&=-f(0)+s\mathscr{L}\{f\}(s)
\end{aligned}
\]
(3)\[\begin{aligned}
\mathscr{L}\{f\pp\}(s)&=s\mathscr{L}\{f\p\}-f\p(0)\\
&=s[s\mathscr{L}\{f\}-f(0)]-f\p(0)\\
&=s^2\mathscr{L}\{f\}-sf(0)-f\p(0)\\
&=s^2F(s)-sf(0)-f\p(0)
\end{aligned}
\]
(4)
\[\begin{aligned}
\mathscr{L}\{e^{at}f\}(s)&=\int_0^\infty e^{-st}e^{at}f(t)\diff t\\
&=\int_0^\infty e^{-(s-a)t}f(t)\diff t\\
&=\mathscr{L}\{f\}(s-a)=F(s-a)
\end{aligned}\]
 \newline
Use laplace transformation to solve nonhomogeneous differential equations
\[ay\pp+by\p+cy=f(t)\dots(*)\qquad\text{a, b, c are constants}
\]
$Y(s)=\mathscr{L}\{y\}(s)\qquad F(s)=\mathscr{L}\{f\}(s)$\\ 
Apply laplace transformation to the both sides of the (*).
\[a\mathscr{L}\{y\pp\}+b\mathscr{L}\{y\p\}+c\mathscr{L}\{y\}=\mathscr{L}\{f\}(s)
\]
\[a[s^2Y(s)-sy(0)-y\p(0)]+b[sY(s)-y(0)]+cY(s)=F(s)
\]
\[Y(s)(as^2+bs+c)-ay(0)s-[ay\p(0)+by(0)]=F(s)\]
\[Y(s)=\frac{ay(0)s+[ay\p(0)+by(0)]+F(s)}{as^2+bs+c}
\]
Apply inverse of the laplace transformation to get the solution of nonhomogeneous solution of (*).
\begin{remark}
The laplace transformation should be a bijection map to make the process rigorous.
\end{remark}
\begin{example}
$\left \{	\begin{gathered}
y\pp-3y\p+2y=e^{3t}	\\
y(0)=1, y\p(0)=0
\end{gathered}	\right.$\\
\[Y(s)=\frac{s-3+\mathscr{L}\{e^{3t}\}(s)}{s^2-3s+2}
\]
\[\begin{aligned}
\mathscr{L}\{e^{3t}\}(s)&=\int_0^\infty e^{-st}e^{3t}\diff t\\
&=\int_0^\infty e^{(3-s)t}\mid_{t=0}^\infty\\
&=-\frac{1}{3-s}=\frac{1}{s-3}
\end{aligned}
\]
\[
\begin{aligned}Y(s)=&\frac{s-3}{(s-1)(s-2)}+\frac{1}{(s-1)(s-2)(s-3)}\\
&=\frac{2}{s-1}+\frac{-1}{s-2}+\frac{\frac{1}{2}}{s-1}+\frac{-1}{s-2}+\frac{\frac{1}{2}}{s-3}\\
&=\frac{5}{2}\frac{1}{s-1}-2\frac{1}{s-2}+\frac{1}{2}\frac{1}{s-3}
\end{aligned}
\]
We want to find $g$ s.t. $\mathscr{L}\{g\}=\frac{1}{s-1}$\\
We can verify $\mathscr{L}\{e^{at}\}=\frac{1}{s-a}$\\
\[\int_0^\infty e^{-st}e^{at}\diff t=\frac{1}{s-a}
\]
Then $y(t)=\frac{5}{2}e^{1-t}-2e^{2t}+\frac{1}{2}e^{3t}$
\end{example}
\paragraph{Uniqueness of laplace transformation}
\begin{proof}
Suppose $f$, $g$ are two piecewise continuous functions. If $\mathscr{L}\{f\}\mathscr{L}\{g\}$ for large $s$, then $f\equiv g$.\\
Sketch\\
\begin{theorem}[Weirstrass Approximation Theorem]
Any continuous function on a interval $[a,b]$ is the uniform limit of a sequence of polynomials. (i.e $\exists$ a sequence of polynomials $\{P_n\}$. s.t. $\max_{a\leq t\leq b}|P_n(t)-f(t)|\rightarrow 0$, as $n\rightarrow\infty$)


\end{theorem}
Let $h=f-g$. Then we have $\mathscr{L}\{f-g\}(s)=0$ for $s$ large enough.\\
Claim:$h\equiv0$\\
For $s$ large, we have $\int_0^\infty e^{-st}h(t)\diff t=0$\\
\[\begin{aligned}&\int_0^\infty e^{-nt}e^{-ct}h(t)\diff t=0\qquad s=n+1+c\text{ Just for construction.}\\
=&\int_0^\infty e^{-nt}(e^{-ct}h(t))e^{-t}\diff t=0\\
=&\int_1^0x^n(e^{-ct}h(t))-\diff x\end{aligned}
\]
$x=e^{-t}$, $\diff x=-e^{-t}\diff t$
\[\begin{aligned}
&\int_0^1x^nq(x)\diff x=0\quad \forall n
\end{aligned}\]
Just sum up some of above equations.
\[\begin{aligned}
&\Rightarrow \int_0^1P_mq(x)\diff x=0\qquad \text{any polynomial} P_m\\
&\Rightarrow \int_0^1q^2(x)\diff x=0\dots(2)\\
&\Rightarrow q\equiv 0\\
&h(t)=0
\end{aligned}
\]
To prove (2), by theorem, pick a seq $P_m\rightarrow q$\\
\[\begin{aligned}|\int P_mq-\int q^2|&=|\int(P_m-q)q|\\
&\leq\int|P_m-q|\cdot|q|\rightarrow0
\end{aligned}
\]
In addition, the it indeed $\rightarrow0$ is because $q$ is also $\rightarrow0$\\
$q(x)=e^{-ct}h(t)$, $h=f-g$ is bounded by a exponential function. Therefore, $c\rightarrow0$, $q(x)\rightarrow0$







\end{proof}





