\chapter{Week5}

\section{Wednesday}\index{Wednesday_lecture}
\subsection{Second order differential equation}
\[F(t,y,y^\prime,y^{\prime\prime})=0
\]
\[y^{\prime\prime}=f(t,y,y^{\prime\prime})
\]
 (*)$\left \{	\begin{gathered}
y^{\prime\prime}+P(t)y^\prime+q(t)y=0	\\
y(t_0)=y_0\\
y^\prime(t_0)=y_0^\prime.
\end{gathered}\right.$ 
\begin{theorem}
Let $P(t)$, $q(t)$ be continuous on $(\alpha,\beta)$ and $t_0\in(\alpha,\beta)$. Then there exists a unique solution for the IVP (*) on the entire interval $(\alpha,\beta)$.
\end{theorem}
\begin{remark}
Remember previous example $y^\prime=1+y^2$, the solution is $y=\tan x$ which makes the above theorem dubious( $p$, $q$ are continuous on entire $\mathbb{R}$, while the solution is only on $(-\frac{\pi}{2},\frac{\pi}{2})$). However, the theorem requires DEs to be linear( $y^\prime=1+y^2$ isn't linear). (PS: linear means the variable only appears with power of 1.)\\
In addition, this theorem will be proved latter.
\end{remark}
\begin{definition}
Define $L[y]=y^{\prime\prime}+P(t)y^\prime+q(t)y$\\ 
The set of all solutions to $L[y]=0$ is denoted by $ker(L)=\{y|L[y]=0\}$

\end{definition}
\begin{proposition}
\quad\\
1.~Ker(L) is a vector space of dimension 2.\\
2.~$\{y_1,y_2\}$ is a basis for ker(L) where\\
 $\left \{	\begin{gathered}
y_1^{\prime\prime}+P(t)y_1^\prime+q(t)y_1=0	\\
y_1(t_0)=1, y_1^\prime(t_0)=0.
\end{gathered}\right.$ \qquad
$\left \{	\begin{gathered}
y_2^{\prime\prime}+P(t)y_2^\prime+q(t)y_2=0	\\
y_2(t_0)=0, y_2^\prime(t_0)=1.
\end{gathered}\right.$ 
\end{proposition}

\begin{proof}
In order to show it's a basis, there are two things that we need to show.
First, any solution $y$ to (*) can be written as a linear combination of $y_1,y_2$. Assume it is $y=c_1y_1+c_2y_2$. What we need to do now is checking whether it satisfies the equation and finding out what $c_1$ and $c_2$ are exactly.
\[\begin{aligned}
L[c_1y_1+c_2y_2]&=(c_1y_1^{\prime\prime}+c_2y_2^{\prime\prime})+p(t)(c_1y_1^\prime+c_2y_2^\prime)+q(t)(c_1y_1+c_2y_2)\\
&=c_1y_1^{\prime\prime}+p(t)c_1y_1^\prime+q(t)c_1y_1+c_2y_2^{\prime\prime}+p(t)c_2y_2^\prime+q(t)c_2y_2\\
&=c_1L[y_1]+c_2L[y_2]=0
\end{aligned}
\]
Therefore, it satisfies the first equation. Now, let's find out the value of coefficient.
\[[c_1y_1+c_2y_2](t_0)=c_1y_1(t_0)+c_2y_2(t_0)=c_1=y(t_0)
\]
\[[c_1y_1+c_2y_2]^\prime(t_0)=c_1y_1^\prime(t_0)+c_2y_2^\prime(t_0)=c_2=y^\prime(t_0)
\]
With $c_1=y(t_0)$, $c_2=y^\prime(t_0)$, $y$ is a solution of (*). In addition, with uniqueness, $y\equiv z=c_1y_1+c_2y_2$.\\
Second, we need to show they are linearly independent.\\
Claim $\{y_1,y_2\}$ are linearly independent, i.e. if there exists constants $c_1$, $c_2$ s.t. $c_1y_1+c_2y_2=0$ $\forall t\in(\alpha,\beta)$ then $c_1=c_2=0$.\\
Let's see why this is the case.
$z=c_1y_1+c_2y_2$ is a solution for (*).
\[z(t_0)=0=c_1y_1(t_0)+c_2y_2(t_0)=c_1+0
\]
\[z^\prime(t_0)=c_1y_1^\prime(t_0)+c_2y_2^\prime(t_0)=c_2=0
\]
\end{proof}
$L[y]=y^{\prime\prime}+P(t)y^\prime+q(t)y=0$\\

$\left \{	\begin{gathered}
y(t_0)=y_0\\
y^\prime(t_0)=Y_0^\prime.
\end{gathered}\right.$ 
Suppose $y_1$, $y_2$ are solutions i.e. $y_1$,$y_2\in$ ker(L).\\
\begin{definition}[Wronskian]$W(t)=y_1(t)y_2^\prime(t)-y_2(t)y_1^\prime(t)$\end{definition}
Lemma:\\
(i)$W^\prime+p(t)W=0$\\
(ii)$W$ is either $\equiv0$ or never 0.\\
\begin{proof}
\[W=y_1y_2^\prime-y_2y_1^\prime
\]
\[\begin{aligned}W^\prime&=y_1^\prime y_2^\prime+y_1y_2^{\prime\prime}-y_2^\prime  y_1^\prime-y_2y_1^{\prime\prime}\\
&=y_1(-py_2^\prime-qy_2)+y_2(py_1^\prime+qy_1)\\
&=-py_1y_2^\prime-y_1qy_2+y_2py_1^\prime+y_2qy_1=p(y_1^\prime y_2-y_1y_2^\prime)=-PW
\end{aligned}
\]
This is $W^\prime+PW=0$.
\[(e^{\int P(t)\diff t}W)^\prime=0
\]
$\Rightarrow$ $\qquad$ $e^{\int P(t)\diff t}W\equiv constant$.\\
Therefore, $W(t)=W(t_0)(e^{-\int_{t_0}^tP(s)\diff s})$
\end{proof}

\begin{proposition}
For any two solutions $y_1$, $y_2$ such that $y_1$, $y_2$ are linearly dependent. $\iff$ $W[y_1,y_2]\equiv0$
\end{proposition}
\begin{proof}
($\Rightarrow$)$y_1$, $y_2$ l.dep\\
$\Rightarrow$ $\exists$ $c_1$, $c_2$ s.t. $c_1y_1+c_2y_2=0$ (not both 0; definition of linearly dependent.)\\
Suppose $c_1\neq0$, $y_1=-\frac{c_2}{c_1}y_2=\delta y_2$\\
$W[y_1,y_2]=y_1y_2^\prime-y_2y_1^\prime=\delta y_2y_2^\prime-y_2\delta y_2^\prime=0$ \qquad ($y_1=\delta y_2$)\\

($\Leftarrow$) \\
This direction we need to consider two different cases. \\
(i)\quad$y_1(t)$, $y_2(t)$ are never 0 in ($\alpha,\beta$)\\
$\Rightarrow$ \quad
$\frac{y_2^\prime}{y_2}=\frac{y_1^\prime}{y_1}$ \quad $\Rightarrow$ $ln|\frac{y_1}{y_2}|=constant$\\
$|\frac{y_1}{y_2}|=Constant$ , i.e. $\frac{y_1}{y_2}=constant$\\
(ii)If $y_1(E)$, $y_2(E)$ =0
 for some $E\in(\alpha,\beta)$, assume $y_1(E)=0$.\\
If $y_1^\prime(E)=0$ $\Rightarrow$ $y_1\equiv0$.\\
If $y_1^\prime(E)\neq0$ \quad$y_1(E)\cdot y_2^\prime(E)=y_2(E)\cdot y_1^\prime(E)$\\
$y_2(E)=\frac{y_2^\prime(E)}{y_1^\prime(E)}y_1(e)=\delta y_1(E)$

$\left \{	\begin{gathered}
y_2, \delta y_1 \textit{ are solutions never 0}\\
y_2(E)=\delta y_1(E)\\
y_2^\prime(E)=\delta y_1^\prime(E).
\end{gathered}\right.$ \\
View $\delta y_1$ as a solution to IVP. By uniqueness, $y_2\equiv\delta y_1$
\end{proof}

As we showed before, $z^\prime=t^2+z^2$ (1) doesn't have a solution that can be expressed by primary functions. Let $z=-\frac{y^\prime}{y}$. 
\[z^\prime=-\frac{y^{\prime\prime}}{y}+\frac{{y^\prime}^2}{y^2}=t^2+\frac{{y^\prime}^2}{y^2}
\]
From right-hand side equation, we get
\[y^{\prime\prime}+t^2y=0 \cdots (2)
\]
Because (1) doesn't have a solution, then (2) equation doesn't have either.\\
